\documentclass[11pt]{article}
\usepackage[utf8]{inputenc}
\usepackage[left=2.00cm,vmargin=1cm]{geometry}

%\usepackage{lstlisting}

%\usepackage{chordbars}

\title{Package chordbars manual}
\author{S. Kramm}

\begin{document}
\maketitle

This package will help produce song structure, showing both bar count and song chords.

It is based on Tikz.

It provides a single environnment named {\ttfamily chordbar}, inside which on can define the chords for each bar.

For example:

\begin{lstlisting}
\begin{chordbar}{4}{1}{Intro}
\chordf{1}{C}
\chordf{2}{D}
\chordf{3}{Em}
\chordf{4}{Em}
\end{chordbar}
\end{lstlisting}

This environment has three arguments
\begin{enumerate}
\item The number of bars needed.
\item The number of times the whole pattern gets repeated.
\item The part name.
\end{enumerate}
\end{document}
