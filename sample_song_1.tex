\documentclass[11pt]{article}
\usepackage[utf8]{inputenc}
\usepackage[left=2.00cm,vmargin=1cm]{geometry}

\usepackage{chordbars2}

\title{Song Sample}
\author{Author name}

\begin{document}
\songtitle

This is a demo example of the chordbars package.

\begin{chordbar}{4}{1}{Intro}
\chordf{C}
\chordf{D}
%\chordh{E}{2}
%\chordh{E}{2}
\chordf{F}
\end{chordbar}

\end{document}

\begin{chordbar}{4}{1}{Intro}
\chordfn{1}{C}
\chordfn{2}{D}
\chordfn{3}{Em}
\chordfn{4}{Em}
\end{chordbar}

% this part has 8 bars (4 bars, 2 lines)
\begin{chordbarl}{8}{4}{3}{2 lines}
\chordf{C}
\chordf{D}
\chordhn{3}{1}{Em}
\end{chordbarl}

% this part has 5 bars on 2 lines
\begin{chordbarh}{5}{2}{5 half bars}
\end{chordbarh}

\begin{chordbarh}{6}{2}{6 half bars}
\end{chordbarh}

\begin{chordbar}{4}{1}{Intro}
\chordf{C}
\chordf{D}
\chordh{Bm}{1}
\chordh{Em}{2}
\chordf{Em}
\end{chordbar}


\begin{chordbar}{4}{1}{This is a very long description}
\chordfn{1}{C}
\chordfn{2}{D}
\chordfn{3}{Em}
\chordhn{4}{Em}{1}
\chordhn{4}{B}{2}
\end{chordbar}

\begin{chordbar}{4}{1}{Intro}
\chordhn{1}{C}{1}
\chordhn{1}{D}{2}
\chordhn{2}{E}{1}
\chordhn{2}{F}{2}
\end{chordbar}


\begin{chordbar}{8}{2}{Verse}
\chordf{Em}
\chordf{Em6}
\chordf{Em7}
\chordf{Em-Maj7}
\chordf{C}
\chordf{C9}
\chordf{C}
\chordf{B}
\end{chordbar}

\begin{chordbar}{8}{1}{Bridge}
\chordf{A7}
\chordf{./.}
\chordf{B7}
\chordf{./.}
\chordf{C}
\chordf{./.}
\chordf{B7}
\chordf{./.}
\end{chordbar}


\end{document}
