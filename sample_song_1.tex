% this is mostly a test file, to check features

\documentclass[11pt]{article}
\usepackage[utf8]{inputenc}
\usepackage[left=2.00cm,vmargin=1cm]{geometry}

\usepackage{chordbars}

\title{Song Sample with a really realllly long name, just to test}
\author{Author name}


\begin{document}
%\bpm{120}
%\bpbthree

\countbarsYes

\songtitle

%\def\barsize{1.5}%	
This is a demo example of the chordbars package.


\begin{chordbar}{4}{Intro}
\chordf{C\flat}
%\chordf{C\flat}
\chordh{A#}{D\sharp}
%\repeatBarPair
\end{chordbar}


Whole line of same chord:

\chordline{C#m7}{8}{Intro}

You can print some notes between the different parts

\def\barsize{1.8}%	

\begin{chordbar}[3]{6}{Intro b}
\chordf{C#}
\repeatBar
\chordh{A}{D}
\chordf{F}
\chordf{F}
\chordf{D}
\end{chordbar}

%\begin{chordbar}{4}{1}{Intro c}
%\chordfn{1}{C}
%%\chordfn{2}{D}
%\chordfn{2}{Em}
%\repeatchord
%\repeatchord
%\end{chordbar}

\def\barsize{1.4}%
% this part has 8 bars (4 bars, 2 lines)
\begin{chordbar}[2]{4}{TEST}
\chordf{C}
\chordf{D}
\chordh{E}{F}
\repeatBar
\end{chordbar}

\begin{chordbar}{5}{Test of repeatbarpair}
\chordh{G}{C7}
\chordf{FMaj7}
\repeatBarPair
\chordf{FMaj7}
\addHalfBar{Dm}
\end{chordbar}

\begin{chordbar}{4}{Test of addHalfBar}
\chordh{G}{C7}
\chordf{FMaj7}
\repeatBar
\chordf{Dm}
\addHalfBar{Gm}
\end{chordbar}


% this part has 8 bars (4 bars, 2 lines, repeated 3 times)
\begin{chordbar}[2]{6}{2 lines}
\chordf{C}
\chordf{D}
\chordh{E}{F}
\repeatBar
\chordf{D}
\repeatBar
%\chordhn{3}{1}{Em}
\end{chordbar}

\begin{chordbar}[4]{11}{2 lines}
\chordf{C}
\chordh{D}{G}
\chordf{E}
%\chordf{F}
\newchordline
\chordf{G\#}
\repeatBar
\end{chordbar}

\begin{chordbar}[2]{10}{2 lines of 5 bars}
\chordf{C}
\chordf{D}
\chordf{E}
\chordf{F}
\newchordline
\chordf{G#}
\chordh{A}{G#}
\repeatBar
\end{chordbar}

%\end{document}
\xdef\NumberOfBarsPerLine{8}
\begin{chordbar}[2]{6}{coda}
\chordf{E}
\chordh{A}{F}
\end{chordbar}


\printNbBars

\end{document}
% this part has 5 bars on 2 lines
\begin{chordbarh}{5}{2}{5 half bars}
%\chordf{C}
\end{chordbarh}

\end{document}

\begin{chordbarh}{6}{2}{6 half bars}
\end{chordbarh}

\begin{chordbar}{4}{1}{Intro}
\chordf{C}
\chordf{D}
\chordh{Bm}{1}
\chordh{Em}{2}
\chordf{Em}
\end{chordbar}


\begin{chordbar}{4}{1}{This is a very long description}
\chordfn{1}{C}
\chordfn{2}{D}
\chordfn{3}{Em}
\chordhn{4}{Em}{1}
\chordhn{4}{B}{2}
\end{chordbar}

\begin{chordbar}{4}{1}{Intro}
\chordhn{1}{C}{1}
\chordhn{1}{D}{2}
\chordhn{2}{E}{1}
\chordhn{2}{F}{2}
\end{chordbar}


\begin{chordbar}{8}{2}{Verse}
\chordf{Em}
\chordf{Em6}
\chordf{Em7}
\chordf{Em-Maj7}
\chordf{C}
\chordf{C9}
\chordf{C}
\chordf{B}
\end{chordbar}

\begin{chordbar}{8}{1}{Bridge}
\chordf{A7}
\chordf{./.}
\chordf{B7}
\chordf{./.}
\chordf{C}
\chordf{./.}
\chordf{B7}
\chordf{./.}
\end{chordbar}

\printNbBars

\end{document}
